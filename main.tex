\documentclass[12pt, a4paper]{article}
\usepackage{caption}
\usepackage{graphicx}
\usepackage{hyperref}
\hypersetup{
    colorlinks,
    citecolor=black,
    filecolor=black,
    linkcolor=black,
    urlcolor=black
}
\usepackage{color}
\usepackage{listings, xcolor}
\lstset{
	language=Java,
	tabsize = 4, 
	showstringspaces = false, 
	numbers = left, 
	commentstyle = \color{purple},
	keywordstyle = \color{blue},
	stringstyle = \color{red}, 
	rulecolor = \color{black},
	basicstyle = \small \ttfamily, 
	breaklines = true,
	numberstyle = \tiny,
}
\usepackage{tikz-network}
\usepackage{amsmath, amsfonts, amssymb, amsthm}
\title{Concurrent programming}
\date{2022}
\author{Kristoffer Klokker}
\begin{document}
	\maketitle
	\clearpage
	\tableofcontents
	\clearpage
	\section{Introduction}
		Concurrency is the act of having multiple execution done simultaneously which interact with each other.\\
		This is done to utilise multiple CPU cores rather than rely on CPU speed.\\
		Not only this but instead of having single powerfull computers, bigger networks of computers can be used.\\
		The benefits comes at a cost of complexity, due to the all possible outcomes of different timed execution.\\
		The risk includes race-conditions where variables are changed while used in another thread or liveness where threads are stuck forever waiting for a non realeased variable nad such.
	\section{Anonymous,- and lambdafunctions}
		For at simple class which is given in an argument, instead of creating a class and then parsing it, the class can be created in the argument field.\\
		For instance a class which implements comparable, it can be programmed as such:
\begin{lstlisting}[language = Java , frame = trBL , firstnumber = last , escapeinside={(*@}{@*)}]
public interface StringExecute {
	public void run(String content);
}
	
public static void doAndMeasure( StringExecutable runnable ) {
	long t1 = System.currentTimeMillis();
	runnable.run();
	System.out.println( "Elapsed time: " + (System.currentTimeMillis() - t1) + "ms" );
}

public static void anonFunc() {
	doAndMeasure(new StringExecute() {
		public static void run(String content) {
			System.out.println(content + " Wow!");
		}
	});
}

public static void lambdaFunc() {
	doAndMeasure( (content) -> System.out.println(content + " Wow!"););
}

public static void lambdaFuncOpt() {
	doAndMeasure( (content) -> System.out::println;); //Only prints content and not + " Wow!"
}
\end{lstlisting}
			Here lambda function is only possible due to the compiler knowing what type of object is created due to restrains from the function and the runnable interface only contains a single run function.
			Another use of lambda expression is when working with maps.
\begin{lstlisting}[language = Java , frame = trBL , firstnumber = last , escapeinside={(*@}{@*)}]
public static void main() {
	String text = "Hello world hope your having a good day!";
	Map<Charachter, Interger> occurrences = new HashMap<>();
	for(int i = 0; i < text.length(); i++){
		final char c = text.charAt(i);
		if(occurrences.containsKey(c))
			occurrences.put(c,occurrences.get(c)+1);
		else
			occurrences.put(c,1);
	}
	for(int i = 0; i < text.length(); i++){
		final char c = text.charAt(i);
		occurrences.merge(c,1,(currValue,value) -> currValue+value);
	}
\end{lstlisting}
		Both for loops do the same, the second uses merge which takes a position (c) and a default value (1) and a bifunction which is a function with two inputs. The arguments will automaticly be assigned such currValue is the current hash value and value is the same as the default value 1.\\
		Some of the most usefull built-in functional interfaces includes:
		\begin{itemize}
			\item Predicates - 1 argument returns boolean
			\item Functions - 1 argument returns result
			\item Suppliers - Like Functions but no arguments
			\item Consumers - 1 argument no return
			\item Comaprators - implements compareTo
		\end{itemize}
	\section{Streams}
		Streams are like an foreach and like the name it is a stream of data. In more fine words it is a monad which is datastructure of a sequence of steps of operations.\\
		There are different types of sterams, the $Stream$ is an object stream, wheras primitives stream also exists $IntStream, LongStream, DoubleStream$ with poosibilites like $IntStream.range(1,4)$ which has a stream of 1,2,3 and \\$IntStream.range(1,4).sum()$\\
		To tell the compiler if an object stream is transformed into a primitive type $.mapTo(Int,Double,...)$ with a parser as argument.\\
		The most common functions used by streams are:
\begin{lstlisting}[language = Java , frame = trBL , firstnumber = last , escapeinside={(*@}{@*)}]
List<String> words = Arrays.asList("watercan", "digital", "citizen");
//Performs function with each element
words.stream().forEach(word -> System.out::println);
//Modifies the element
words.stream().map(word -> String::length);
//Only elements which fullfill the function will 'parse'
words.stream().filter(word -> word.startsWith("d"));
//Streams the sorted stream
words.stream().sorted((s1, s2) -> s1.compareTo(s2)); 
//Counts the number of elements
words.stream().count();
//replace element by stream of given function
words.stream().flatmap(word -> Stream.of(word.split("a")); 
//Gather elements in stream, for analyzis or into an object ex list
words.stream().collect(Collectors.toList());
//Reduce takes a stream and makes a single element Ex. this takes the longest string
words.stream().reduce((s1,s2) -> s1.length() > s2.length() ? s1 : s2);
\end{lstlisting}
		The functions can then be chaineed together, so methods which returns a stream can be worked upon.\\
		When working with file stream it is done as following:
\begin{lstlisting}[language = Java , frame = trBL , firstnumber = last , escapeinside={(*@}{@*)}]
try( Stream< String > lines = Files.lines(Paths.get("text.txt"))) {
	lines.forEach(System.out::println);
}
catch ( IOException e) {
	e.printStackTrace();
}
\end{lstlisting}
		By using the try and putting it into the parenthesis it will cause the stream to close when the stream is finished.\\
		When a terminal operation (returns result rather than intermideate which returns stream) is done one the stream, it will close the stream. To counteract this a function which returns the stream may be used as such:
	\begin{lstlisting}[language = Java , frame = trBL , firstnumber = last , escapeinside={(*@}{@*)}]
Supplier<Stream<String>> streamSupplier =  () -> Stream.of("d2", "a2").filter(s -> s.startsWith("a"));
//Returns true due to there is elements in the stream
streamSupplier.get().anyMatch(s -> true);  //
//Returns false due to elements existing in stream
streamSupplier.get().noneMatch(s -> true); 
\end{lstlisting}	
	\subsection{Parrallel streams}
		When working with parrallel stream the stream is outsourced to threads using a common ForJoinPool.\\
		This makes stream elements being handled side by side.\\
		Some function needs modifications such as reduce:
\begin{lstlisting}[language = Java , frame = trBL , firstnumber = last , escapeinside={(*@}{@*)}]
words.stream().parallelStream().reduce(0,(sum,s) -> sum + s.length(), (sum1, sum2) -> sum1+sum2);
\end{lstlisting}		
	Reduce is now broken up onto three arguments,
	\begin{itemize}
		\item indetifier - initial value
		\item accumulator - takes current result and element and operatoes
		\item Combiner - takes two partial accumulators and combines them
	\end{itemize}
	Like sot the $sort()$ function will either wait for all parralle streams or use parralell sorting in large amounts.
\section{Bugs and solutions of parallelizing}
	When working with parrallel threads it is important to encapsulate as much as possible. Threads may access data outside it thread, but this can lead to some of the following bugs.
	\subsection{Race conditions}
		This is a bug which accour when two threads are accessing the same global variable.\\
		This is a bug based upon atomicity is not kept and an operation is not done before another access data from the operation.\\
		The most simple form is two threads which are incrementing a variable.\\
		Say that the first threads read the variable, at the same the second thread reads the variable.\\
		This will conclude in both threads having the same value for the variable and when incrementing the variable will only increment once.\\
		To counteract this lock can be used. This is where code snippets are ran syncronised and threads are not able to execute twice.
	\subsection{Unmutable objects}
		Most often objects are a problem of race conditions.\\
		To counteract this an object is made unmutable.\\
		This means the object:
		\begin{itemize}
			\item is not including setters
			\item is not including references to itself
			\item has only final methods and variables
			\item is private such no method is overwritten
		\end{itemize}
	\subsection{Low level thread locking}
		The low level method for locking threads used in libraries uses an object and it ability to wait() and notify().\\
		\begin{lstlisting}[language = Java , frame = trBL , firstnumber = last , escapeinside={(*@}{@*)}]
synchronized(monitor) {
	try {
		monitor.wait();
	} catch(InterruptedException e) e.printStackTrace();
}
.
.
.
synchronized(monitor) { monitor.notify();}
		\end{lstlisting}
		This code will in one thread use the try catch which will wait for the object called monitor. Then in another thread the monitor is notified.\\
		The problem is if the monitor is notified before the other thread is waiting the thread will never escape the wait.\\
		So another boolean is needed such after a notify it is changed and the try is in an if to the boolean.
\section{Threading in Java}
	Threading is done thorugh the $Thread$ object which takes a runable object as an argument.\\
	The thread is then executed by $start()$ and $join()$ is used as a wait till a thread is done.\\
	If join is not used the main thread will stop its execution before child threads are done.\\
\begin{lstlisting}[language = Java , frame = trBL , firstnumber = last , escapeinside={(*@}{@*)}]
Thread t1 = new Thread( () -> {for(int i = 0; i < 1000; {i++; System.out.println(i);})});
t1.start();
try{
	t1.join();
}
catch(InterruptedException e) {e.printStackTrace();}
\end{lstlisting}
	The try is due to threads may be interrupted, which otherwise would case the program to halt, due to the thread never joining.\\

	\subsection{Syncronize}
		In java the syncronize keyword is used to lock objects or methods, such race conditions are prevented.\\
		An exampel of the use of syncronization can be seen below.
\begin{lstlisting}[language = Java , frame = trBL , firstnumber = last , escapeinside={(*@}{@*)}]
private static class Counter {
	private int c = 0;
}

public static void main() {
	Thread t1 = new Thread( () -> {for(int i = 0; i < 1000; i++) {
		synchronized( counter ) {
			counter.c++;
		}
	});
	t1.start();
	Thread t2 = new Thread( () -> {for(int i = 0; i < 1000; i++) {
		synchronized( counter ) {
			counter.c++;
		}
	});
	t2.start();
	try{
		t1.join();
		t2.join();
	}
	catch(InterruptedException e) {e.printStackTrace();}
\end{lstlisting}	
	The synchronized will here lock the counter object before access from which the code is executed. This will ensure that c is equal 2000.\\
	Another way is to making a whole method syncronised by:
	\begin{lstlisting}[language = Java , frame = trBL , firstnumber = last , escapeinside={(*@}{@*)}]
private synchronized void increment() {
	 i++;
}
\end{lstlisting}
		\subsection{Spin lock}
			Spin locks are an alternative to syncronizing.\\
			The way it works, is a global atomic boolean variable is set, and used as being true if the object is locked.\\
			Then instead of writing syncronized it can be written as followed:
	\begin{lstlisting}[language = Java , frame = trBL , firstnumber = last , escapeinside={(*@}{@*)}]
Thread t = new Thread( () -> {for(int i = 0; i < 1000; i++) {
	while(!locked.compareAndSet(false,true)){}
	counter.c++;
	locked.set(true);
	}
});
}
\end{lstlisting}
			Here compareAndSet is an atomic function which compares to the second argument an returns true if equal. If not then it reutrns false and set the value equal to the first argument.\\
			
		\subsection{Latches}
			When working with many threads, instead of using each threads join method, a CountDownLatch can be used.\\
			The latch is initialized with the amount of threads which will be used: CountDownLatch latch = new CountDownLatch(10);\\
			Then every thread should end with: latch.countDown();\\
			Because then when joining the following can be done:
	\begin{lstlisting}[language = Java , frame = trBL , firstnumber = last , escapeinside={(*@}{@*)}]
	try{
		latch.await();
	}
	catch(InterruptedException e) {e.printStackTrace();}
\end{lstlisting}
			Latches may also be used, to wait for all threads to be initialized before starting, in case of benchmark or such.\\
		\subsection{Future task}
			A future task object, is a threading object, which takes a task.\\
			The object can be called with $.get()$ which will return the result of the task.\\
			When the task is done, the object stays in the finished state.\\
			This can be used in case of a cache, where earlier results are in a hash table in form of future tasks.
		\subsection{Semaphores}
			Semaphores are objects, which is used to manage the amount of threads in a pool.\\
			It does this using permits, which are given to a thread to start a task.\\
			When the task is done the permit is given back to the semaphore.\\
			These are very usefull in cases of API's which has a maximum amount of connections.
		\subsection{Barriers}
			Barriers are a lot like lateches.\\
			These work to ensure every thread is at the same point.\\
			This may be used in simulations, where data from another thread is needed, and it is needed to know that the data is already computed or at the same point in time.
		\subsection{Iterating over safe objects}
			Iterating over a thread safe object may lead to problems. The most common method of to simply try iterate and in case of a change in the collection a \textit{ConcurrentModificationException} will be thrown.\\
			There is other solutions including locking the obejct while iterating or creating a clone of the object and iterator throught it.\\
			Remember to be carefull of hidden iterations, this can be things like $toString()$ which iterates through some object and converts it into chars for a string.
		\subsection{Thread safe classes}
			Threads safe classes are a good way to make code thread safe. These classes includes ConcurrentHashMap and CopyOnWriteArrayList.\\
			These classes does not eliminate concurrency errors, self made check then write errors can still accour, but the classes include function for thos purposes.
		\subsection{Using queues}
			Queues provide a method for using the producer consumer pattern.\\
			This pattern uses most often block queues to store task, created by producer which consumers can perform when possible.\\
			The pattern also helps making code more readable, by dividing it into creating task and performing tasks.\\
			Queues also handles throttling due to it being able to handle a maximum amount of tasks which is being asked for at once.\\
			The queue acts as a middleman for storing the tasks and ensures ownership of objects are changed.\\
			In case of the consumer also being a producer a deque (pronounced 'deck') is used, which is a queue with fast accessing both head and tail, to insert and remove tasks.\\
			This can be implemented as:
	\begin{lstlisting}[language = Java , frame = trBL , firstnumber = last , escapeinside={(*@}{@*)}]
int maxThreads = Runtime.getRuntime().availableProcessors() -1;
CountdownLatch latch = new CountDownLatch(maxThreads);
final BlockingDeque<Optional<Path>> tasks = new LinkedBlockingDeque<>();
IntStream.range(0,maxThreads).forEach( i -> {
	new Thread(() -> {
		try {
			Optional<Path> task;
			do {
				task = tasks.take();
				task.ifPresent( object -> function(object));
			} while(task.isPresent());
			tasks.add(task);
		} catch(InterruptedException e) {}
		latch.countDown();
	}).start();
}
\end{lstlisting}
			This code gets the number of avaliable threads minus the current thread, the creates a queue and latch.\\
			Then threads are created according to the amount of avaliable threads, which takes task from the queue until an empty Optional.\\
			It adds it again so other threads also can stop.\\
			The reason the use a pool of threads is due to the time makign threads and too many threads needs switching between by the CPU if there are more threads than cores.\\
		\subsection{Executor framework}
			The executor framework is a framework to work with pools of threads.\\
			It implements a number of threads and a task queue.
	\begin{lstlisting}[language = Java , frame = trBL , firstnumber = last , escapeinside={(*@}{@*)}]
private static final Executor exec = Executors.newFixedThreadPool(n);
exec.execute(task);
\end{lstlisting}
			This code creates a form of executor with a fixed number of threads, from which execute on runnable called task.\\
			The framework also support ScheduledThreadPoolExecutor for timed tasks which should be used over the timer class.
		\subsection{Thread optimization}
			To use threads efficiently are states used. These stated includes BLOCKED, WAITING or TIMED$\_$WAITING.\\
			By using these states threads, can be interrupted and used for another task meanwhile waiting or being blocked.\\
			A thread can also be interrupted by another thread. Though the thread will continue until a point which makes sense, which may be when it is done.
	\section{Safe shared objects}
		Shared objects between threads can introduce many dangers, mainly that thread can be sure to read the same value it just wrote.\\
		Stale data is term for none up to date values from shared objects which can lead to non expected outcomes.\\
		A solution for this is syncronizing the shared object. By this only one thread can access the shared object at a time, therefore ensuring no stale data.\\
		Another solution is all variables being local to each threat\\[4mm]
		In Java there is a safety gurantee to variables, that once updated they are fully dated. Unless the type of variable is long or double, due to the 64 bit required data, it is updated by two times of 32 bit each, therefore half updated values may accour.\\
		Syncronization is done by locking the given object, such no other thread can access it. \\[4mm]
		Volatile variables are a weak syncronised variable, due to not ensuring atomicity. It is not as safe, but can be used for veryfying states rather correctness.\\
		Therefore volatile variables should only be used when its value does not depend on current value or only one threads updates aswell as the variable not participating in invariants.\\[4mm]
		Ad-hoc thread confinement is a way of describing how an object may only be in certain safe ways. This is ofcourse not a stable way of implementing objects.\\
		A more stable alternative is stack confinement, where reference an variables are all local to a thread.\\
		ThreadLocal solves helps to ensure no leaks could accour in stack confiment, by making an object with variables which can not be accessed outside a thread.\\
		Another way is an immutable object, which is a final object which only contains final fields, can not be modified and construction has not been escaped.\\
		It is not enough to just make the object reference final, due to the object itself being able to change its variables.\\
		The immutable object may also be good in case of outside methods, this ensures that the methods may not change anything or save a reference.\\
		A object may be treated as ummutable such as the date object, but it requires clear documentation.\\
		
		\subsection{Constructing objects}
			When constructing objects it is important, that a reference to the object not escape before initialization.\\
			This can result in half initialized objects being treated as done.\\
			To prevent this a static initializer may be used, which creates the object, parses it to a method which makes it public and returns the reference.
		\subsection{Thread safe classes}
			First the invariant variables are identified and variables for object states. Such they can get policy according to use.\\
			State-depend operations are preconditions on operations, which also acts as a policy for the class.\\
			It will always be easier to encapsulate methods for objects such syncronised always has the correct lock.\\
			Remember to always use thread safe classes, to make life easier but still remember to handle them safely.\\
			When access multiple variables wich relates send them as tuples, such they always are syncronized.\\
			A thread safe class may include public variables, which has to be handled safe but in some cases it makes sense to make them public.\\
			Some class may have to be extended with a syncronised method, such as a list extension with a method for adding of absent.\\
			A custom lock may also be but in a method Ex. public boolean m() \{ synchronzed(list) \{...\}\}, use the list variable as a custom lock.\\
			
\end{document}
