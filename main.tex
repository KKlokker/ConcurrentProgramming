\documentclass[12pt, a4paper]{article}
\usepackage{caption}
\usepackage{graphicx}
\usepackage{hyperref}
\hypersetup{
    colorlinks,
    citecolor=black,
    filecolor=black,
    linkcolor=black,
    urlcolor=black
}
\usepackage{listings}
\usepackage{tikz-network}
\usepackage{amsmath, amsfonts, amssymb, amsthm}
\title{Concurrent programming}
\date{2022}
\author{Kristoffer Klokker}
\begin{document}
	\maketitle
	\clearpage
	\tableofcontents
	\clearpage
	\section{Introduction}
		Concurrency is the act of having multiple execution done simultaneously which interact with each other.\\
		This is done to utilise multiple CPU cores rather than rely on CPU speed.\\
		Not only this but instead of having single powerfull computers, bigger networks of computers can be used.\\
		The benefits comes at a cost of complexity, due to the all possible outcomes of different timed execution.
		
\end{document}